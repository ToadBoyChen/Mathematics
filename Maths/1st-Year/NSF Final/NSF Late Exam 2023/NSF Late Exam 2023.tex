\documentclass{article}
\usepackage{amsmath}
\usepackage{amssymb}
\usepackage{geometry}
\usepackage{microtype}
\hbadness=99999
\ProvidesPackage{mymathsymbols}[2024/07/28 A custom package for common mathematical symbols]

\RequirePackage{amsfonts} % This package is needed for \mathbb
\RequirePackage{pifont}   % This package is needed for \ding

% Define custom commands
\newcommand{\xmark}{\text{\ding{55}}}
\newcommand{\R}{\ensuremath{\mathbb{R}}}
\newcommand{\C}{\ensuremath{\mathbb{C}}}
\newcommand{\Z}{\ensuremath{\mathbb{Z}}}
\newcommand{\Q}{\mathbb{Q}}
\newcommand{\N}{\mathbb{N}}
\newcommand{\F}{\mathbb{F}}
\newcommand{\W}{\mathbb{W}}
\newcommand{\all}{~\forall~}
\newcommand{\power}{\mathcal{P}}

\endinput %Download the macros file locally onto machine.
\usepackage{array}
\usepackage{hyperref}
\hypersetup{
    colorlinks,
    citecolor=black,
    filecolor=black,
    linkcolor=black,
    urlcolor=black
}
\geometry{a4paper, margin=1.6in}

\title{NSF Late Summer Exam 2023}
\author{by Toby Chen}
\date{\today}

\begin{document}

    \maketitle

    \tableofcontents

    \section{Question 1}
        Let $A = \left\{ 1,3,5,6,8 \right\}$, and let $B = \left\{ 2,3,4,8,9 \right\}$ 

        \sub{Q1 a}
            $A\cap B$.\\

            \ans $A\cap B = \left\{ 3, 8 \right\}$

        \sub{Q1 b}
            $\left\{ b + 2: b\in B \AND b-2 \notin A \right\}$.\\

            \ans $\left\{ 4, 6, 11 \right\}$

        \sub{Q1 c}
            The number of 3 element subsets of $B$.\\

            \ans For each element, there are 3 possible arrangements (not including order - order is of no concern), and in $B$ there are 5 elements so therefore $3\times 5 = 15$. \xmark\\

            \rans The problem with my method is that the number of possible sets decrease as we move along,

            \begin{align*}
                \text{Given }&B=\left\{ 2,3,4,8,9 \right\},\\
                \text{Taking }2\text{ first:}&\left\{ 2,3,4 \right\}, \left\{ 2,4,8 \right\}, \left\{ 2,8,9 \right\}, \left\{ 2, 4, 9 \right\}, \left\{ 2,3,9 \right\}, \left\{ 2,3,8 \right\},\\
                \text{Taking }3\text{ first:}&\left\{ 3,4,8 \right\}, \left\{ 3,8,9 \right\}, \left\{ 3,4,9 \right\},\\
                \text{Taking }4\text{ first:}&\left\{ 4,8,9 \right\}.\\
                &\text{10 3-elemented sets possible}.\\
            \end{align*}

        \sub{Q1 d}
            The number of proper subsets of $B$.\\

            \ans The number of proper subsets is also $\power(B)$. There are 5 1-elemented sets, 20 2-elemented sets, 15 3-elemented sets, 10 4-elemented subsets, so therefore $\power(B) = 5 + 20 + 15 + 10 = 60.$ \xmark\\

            \rans

\end{document}