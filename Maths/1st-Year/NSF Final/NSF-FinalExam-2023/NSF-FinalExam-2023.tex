\documentclass{article}
\usepackage{amsmath}
\usepackage{amssymb}
\usepackage{geometry}
\ProvidesPackage{mymathsymbols}[2024/07/28 A custom package for common mathematical symbols]

\RequirePackage{amsfonts} % This package is needed for \mathbb
\RequirePackage{pifont}   % This package is needed for \ding

% Define custom commands
\newcommand{\xmark}{\text{\ding{55}}}
\newcommand{\R}{\ensuremath{\mathbb{R}}}
\newcommand{\C}{\ensuremath{\mathbb{C}}}
\newcommand{\Z}{\ensuremath{\mathbb{Z}}}
\newcommand{\Q}{\mathbb{Q}}
\newcommand{\N}{\mathbb{N}}
\newcommand{\F}{\mathbb{F}}
\newcommand{\W}{\mathbb{W}}
\newcommand{\all}{~\forall~}
\newcommand{\power}{\mathcal{P}}

\endinput %Download the macros file locally onto machine.
\usepackage{array}
\usepackage{hyperref}
\hypersetup{
    colorlinks,
    citecolor=black,
    filecolor=black,
    linkcolor=black,
    urlcolor=black
}
\geometry{a4paper, margin=1in}

\title{Number, Sets and Functions 2023 Exam}
\author{by Toby Chen}
\date{\today}

\begin{document}

    \maketitle

    \tableofcontents

    \section{Question One}
        Let $X = \{1,3,4,6,9\}$ and let $Y = \{2,3,5,8,9\}$.\\ Write down each of the following sets. \textit{No justification is needed},

        \begin{align*}
            &\text{(a) } X\cup Y,\\
            &\text{answer: } \{1,2,3,4,5,6,8,9\}. &\checkmark\\
            &\\
            &\text{(b) }X\triangle Y,\\
            &\text{answer: } \{1,2,4,5,6,8\}. &\checkmark\\
            &\\
            &\text{(c) } \{X\in x: x+2 \notin X\},\\
            &\text{answer: } \{3,6,9\}. &\checkmark\\
            &\\
            &\text{(d) } \{y+2 : y\in Y \land y-2 \in X\},\\
            &\text{answer: } \{5,7,10\}. &\checkmark\\
        \end{align*}

        Write down the supremum of each of the following sets

        \begin{align*}
            &\text{(e) } \{x^2 : -2 \leq x \leq 1\},\\
            &\text{answer: } 1.&\xmark\\
            &\\
            &\quad\text{Real answer is }4, \text{ Simply square }-2.\\
            &\\
            &\text{(f) } \left\{\frac{n}{n+1}: n \in \mathbb{N}\right\},\\
            &\text{answer: } \frac{n}{n+1} = \frac{1}{1 + \frac{1}{n}},\\
            &\quad\lim_{n\rightarrow \infty}\frac{1}{1 + \frac{1}{n}} = \frac{1}{1 + 0} = 1.&\checkmark\\
            &\\
            &\text{(g) } \{\sin(x) : x\in \mathbb{Q}\},\\
            &\text{answer: }1. &\checkmark\\
        \end{align*}

    \section{Question Two}
        \begin{align*}
            &\text{(a) Define precisely what it means for a function }f:A\rightarrow B\text{ to be injective}.\\
            &\text{answer: Injective means that both }A\text{ and }B \text{ are the same, i.e., the domain and codomain are the same}.&\xmark\\
            &\\
            &\quad\text{Real answer: }\forall~ a,b\in A\text{ if }f(a) = f(b) \implies a = b.\\
            &\\
            &\text{(b) Define precisely what it means for a function }f:A\rightarrow B\text{ to be surjective}.\\
            &\text{answer: Surjective means ?}&\xmark\\
            &\\
            &\\
            &\quad\text{Real answer: }\forall~ b\in B ~\exists~a\in A : f(a) = b.\\
            &\\
            &\text{Are the following injective or not?}\\
            &\text{(c) }f:\mathbb{Z}\rightarrow\mathbb{Z}, f(n) = 20n + 22.\\
            &\text{answer: given our definition of injectivity, }\forall~ m,n\in \mathbb{Z}, f(m) = 20m+22 = 20n + 22 = f(n) \implies m = n.\\
            &f:\mathbb{Z}\rightarrow\mathbb{Z}, f(n) = 20n + 22 \text{ is injective.} & \checkmark\\
            &\\
            &\text{(d) }f:\Z\rightarrow\Z, f(n) = n(n+1).\\
            &\text{answer: Given that }\all a,b\in A\text{ if }f(a) = f(b) \implies a = b,\\
            &\text{Let }n, m\in \Z : f(n) = f(m) \equiv n(n+1) = m(m+1) \equiv m = n \implies f\text{ is injective}.&\xmark\\
            &\\
            &\quad\text{Real answer: counter proof by counter example, }f(-1) = 0 = f(0) = 0(0 + 1) = 0.\\
            &\\
            &\text{(e) }f:\power(\Q) \rightarrow \power(\Q), f(a) = a\cup\{1,2,3\}\\
            &\text{answer: First lets understand the function. We are working within the power series of the rationals.}\\
            &\text{This means that we are working with sets and not numbers. }\\
            &f(A) = A\cup\{1,2,3\}\text{ takes the intersection of two sets within }\Q.\\
            &\text{Recalling that for }f\text{ to be injective, }\all a,b\in A \text{ if }f(a) = f(b) \implies a = b,\\
            &\text{Let }n = \{1\}, m = \{2\}: f(n) = \{1,2,3\}, f(m) = \{1,2,3\} \not{\implies} n = m\text{ as } \{1\} \neq \{2\}.\\
            &\text{This suggests that }f \text{ is not injective.} &\checkmark\\
            &\\
            &\text{(f) }f:\N\times\N\rightarrow\Z\times\Z, f(m,n) = (m^2 + n^2, m^2 - n^2).\\
            &\text{answer: Given }\all a,b\in A \text{ if }f(a) = f(b) \implies a = b,\\
            &f(1,1) = (2, 0), f(-1, -1) = (2, 0) \text{ but } (1,1)\neq (-1,-1).\\
            &\text{By such logic, }f \text{ is not injective.}&\xmark\\
            &\\
            &\quad\text{Real answer: Notice the domain is }\N\times\N\text{ which means that }(-1,-1)\notin \N\times\N.\\
            &\quad\text{This is the mistake made. If we notice any input is a positive integer, the proof follows, }\\
            &\quad n,m, p, q \in \N\times\N : f(n,m) = f(p,q).\\
            &\quad \implies m^2 + n^2 = p^2 + q^2 \text{ and }m^2 - n^2 = p^2 - q^2.\\
            &\quad m^2 = p^2 - q^2 + n^2 \implies p^2 - q^2 + n^2 + n^2 = p^2 + q^2 \implies n^2 = q^2 \text{ or } n = q.\\
            &\quad \text{Given }n = q, m^2 = p^2 \implies m = q.\text{ Index positions match, and we are only concerned with positive inputs.}\\
            &\quad\text{Given }m = p, n = q \impliedby f(n,m) = f(p,q),~ f\text{ is injective.}\\
        \end{align*}

    \section{Question Three}
        \begin{align*}
            &\\
        \end{align*}

\end{document}
