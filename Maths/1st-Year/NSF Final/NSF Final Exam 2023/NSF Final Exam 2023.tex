\documentclass{article}
\usepackage{amsmath}
\usepackage{amssymb}
\usepackage{geometry}
\usepackage{microtype}
\hbadness=99999
\ProvidesPackage{mymathsymbols}[2024/07/28 A custom package for common mathematical symbols]

\RequirePackage{amsfonts} % This package is needed for \mathbb
\RequirePackage{pifont}   % This package is needed for \ding

% Define custom commands
\newcommand{\xmark}{\text{\ding{55}}}
\newcommand{\R}{\ensuremath{\mathbb{R}}}
\newcommand{\C}{\ensuremath{\mathbb{C}}}
\newcommand{\Z}{\ensuremath{\mathbb{Z}}}
\newcommand{\Q}{\mathbb{Q}}
\newcommand{\N}{\mathbb{N}}
\newcommand{\F}{\mathbb{F}}
\newcommand{\W}{\mathbb{W}}
\newcommand{\all}{~\forall~}
\newcommand{\power}{\mathcal{P}}

\endinput %Download the macros file locally onto machine.
\usepackage{array}
\usepackage{hyperref}
\hypersetup{
    colorlinks,
    citecolor=black,
    filecolor=black,
    linkcolor=black,
    urlcolor=black
}
\geometry{a4paper, margin=1.6in}

\title{Number, Sets and Functions 2023 Exam}
\author{by Toby Chen}
\date{\today}

\begin{document}

    \maketitle

    \tableofcontents

    \section{Question One}
        Let X = \{1,3,4,6,9\} and let Y = \{2,3,5,8,9\}.\\ Write down each of the following sets. \textit{No justification is needed},

        \begin{align*}
            &\text{(a) } X\cup Y,\\
            &\text{answer: } \{1,2,3,4,5,6,8,9\}. &\checkmark\\
            &\\
            &\text{(b) }X\triangle Y,\\
            &\text{answer: } \{1,2,4,5,6,8\}. &\checkmark\\
            &\\
            &\text{(c) } \{X\in x: x+2 \notin X\},\\
            &\text{answer: } \{3,6,9\}. &\checkmark\\
            &\\
            &\text{(d) } \{y+2 : y\in Y \land y-2 \in X\},\\
            &\text{answer: } \{5,7,10\}. &\checkmark\\
        \end{align*}

        Write down the supremum of each of the following sets

        \begin{align*}
            &\text{(e) } \{x^2 : -2 \leq x \leq 1\},\\
            &\text{answer: } 1.&\xmark\\
            &\\
            &\quad\text{Real answer is }4, \text{ Simply square }-2.\\
            &\\
            &\text{(f) } \left\{\frac{n}{n+1}: n \in \mathbb{N}\right\},\\
            &\text{answer: } \frac{n}{n+1} = \frac{1}{1 + \frac{1}{n}},\\
            &\quad\lim_{n\rightarrow \infty}\frac{1}{1 + \frac{1}{n}} = \frac{1}{1 + 0} = 1.&\checkmark\\
            &\\
            &\text{(g) } \{\sin(x) : x\in \mathbb{Q}\},\\
            &\text{answer: }1. &\checkmark\\
        \end{align*}

    \section{Question Two}
        \begin{align*}
            &\text{(a) Define precisely what it means for a function }f:A\rightarrow B\text{ to be injective}.\\
            &\text{answer: Injective means that both }A\text{ and }B \text{ are the same, i.e., the domain and codomain are the same}.&\xmark\\
            &\\
            &\quad\text{Real answer: }\forall~ a,b\in A\text{ if }f(a) = f(b) \implies a = b.\\
            &\\
            &\text{(b) Define precisely what it means for a function }f:A\rightarrow B\text{ to be surjective}.\\
            &\text{answer: Surjective means ?}&\xmark\\
            &\\
            &\\
            &\quad\text{Real answer: }\forall~ b\in B ~\exists~a\in A : f(a) = b.\\
            &\\
            &\text{Are the following injective or not?}\\
            &\text{(c) }f:\mathbb{Z}\rightarrow\mathbb{Z}, f(n) = 20n + 22.\\
            &\text{answer: given our definition of injectivity, }\forall~ m,n\in \mathbb{Z}, f(m) = 20m+22 = 20n + 22 = f(n) \implies m = n.\\
            &f:\mathbb{Z}\rightarrow\mathbb{Z}, f(n) = 20n + 22 \text{ is injective.} & \checkmark\\
            &\\
            &\text{(d) }f:\Z\rightarrow\Z, f(n) = n(n+1).\\
            &\text{answer: Given that }\all a,b\in A\text{ if }f(a) = f(b) \implies a = b,\\
            &\text{Let }n, m\in \Z : f(n) = f(m) \equiv n(n+1) = m(m+1) \equiv m = n \implies f\text{ is injective}.&\xmark\\
            &\\
            &\quad\text{Real answer: counter proof by counter example, }f(-1) = 0 = f(0) = 0(0 + 1) = 0.\\
            &\\
            &\text{(e) }f:\power(\Q) \rightarrow \power(\Q), f(a) = a\cup\{1,2,3\}\\
            &\text{answer: First lets understand the function. We are working within the power series of the rationals.}\\
            &\text{This means that we are working with sets and not numbers. }\\
            &f(A) = A\cup\{1,2,3\}\text{ takes the intersection of two sets within }\Q.\\
            &\text{Recalling that for }f\text{ to be injective, }\all a,b\in A \text{ if }f(a) = f(b) \implies a = b,\\
            &\text{Let }n = \{1\}, m = \{2\}: f(n) = \{1,2,3\}, f(m) = \{1,2,3\} \not{\implies} n = m\text{ as } \{1\} \neq \{2\}.\\
            &\text{This suggests that }f \text{ is not injective.} &\checkmark\\
            &\\
            &\text{(f) }f:\N\times\N\rightarrow\Z\times\Z, f(m,n) = (m^2 + n^2, m^2 - n^2).\\
            &\text{answer: Given }\all a,b\in A \text{ if }f(a) = f(b) \implies a = b,\\
            &f(1,1) = (2, 0), f(-1, -1) = (2, 0) \text{ but } (1,1)\neq (-1,-1).\\
            &\text{By such logic, }f \text{ is not injective.}&\xmark\\
            &\\
            &\quad\text{Real answer: Notice the domain is }\N\times\N\text{ which means that }(-1,-1)\notin \N\times\N.\\
            &\quad\text{This is the mistake made. If we notice any input is a positive integer, the proof follows, }\\
            &\quad n,m, p, q \in \N\times\N : f(n,m) = f(p,q).\\
            &\quad \implies m^2 + n^2 = p^2 + q^2 \text{ and }m^2 - n^2 = p^2 - q^2.\\
            &\quad m^2 = p^2 - q^2 + n^2 \implies p^2 - q^2 + n^2 + n^2 = p^2 + q^2 \implies n^2 = q^2 \text{ or } n = q.\\
            &\quad \text{Given }n = q, m^2 = p^2 \implies m = q.\text{ Index positions match, and we are only concerned with positive inputs.}\\
            &\quad\text{Given }m = p, n = q \impliedby f(n,m) = f(p,q),~ f\text{ is injective.}\\
        \end{align*}

    \section{Question Three}
        \begin{align*}
            &\\
            &\text{(a) Suppose }P, Q, R\text{ are statements. Complete the following truth table for the statement }P\implies Q\land Q\implies \bar{R}.\\
            &\text{answer: }\\
        \end{align*}

        \begin{center}
            \begin{tabular}{|c|c|c|c|}
                \hline
                $P$ & $Q$ & $R$ & $(P\implies Q)\land (Q \implies \bar{R})$\\
                \hline
                t & t & t & f\\
                t & t & f & \underbar{f}\\
                t & f & t & \underbar{f}\\
                t & f & f & f\\
                f & t & t & f\\
                f & t & f & \underbar{t}\\
                f & f & t & \underbar{t}\\
                f & f & f & t\\
                \hline
            \end{tabular} \xmark
        \end{center}

        Real answer: 

        \begin{center}
            \begin{tabular}{|c|c|c|c|}
                \hline
                $P$ & $Q$ & $R$ & $(P\implies Q)\land (Q \implies \bar{R})$\\
                \hline
                t & t & t & f\\
                t & t & f & \underbar{t}\\
                t & f & t & f\\
                t & f & f & f\\
                f & t & t & f\\
                f & t & f & t\\
                f & f & t & t\\
                f & f & f & t\\
                \hline
            \end{tabular}
        \end{center}trivial mistake only.

        \begin{align*}
            &\\
            &\text{(b) Suppose }x, y, z \in \R. \text{ Write down the contrapositive of the following,}\\
            &x^2 > y^2 \implies ~\exists~ w\in\R: x < w \lor w < z.\\
            &\text{answer: }\all w \in \R : w \leq x \land w \geq z \implies x^2 \leq y^2&\checkmark\\
            &\\
            &\quad\text{There is a better way to state this, }z \leq w \leq x \all w \in \R \implies x^2 \leq y^2 \\
            &\\
            &\text{(c) Define the sequence }a_1, a_2, a_2, \dots\text{ of integers by }\\
            &a_1 = 0, a_n = 4a_{n-1} + 12 \text{ for } n \geq 2.\\
            &\text{Prove by induction that }a_n = 4^n - 4 \all n \in \N.\\
            &\text{answer: Let us recall what proof by induction is. First we state our base case. This is }P(1).\\
            &\text{Let us also remember that }P(n): a_n = 4^n - 4 \all n\in\N.\\
            &\text{Our inductive hypothesis is: Given }P(1)\text{ holds, show that for }n \geq 2, \text{ if }P(n-1) \text{ holds, then } P(n) \text{ also holds.}\\
            &\text{We therefore need to prove }P(n-1).\\
            &\\
            &P(n-1): a_{n-1} = 4^{n-1} - 4 \all n\in \N. \text{ Recall } a_n = 4(a_{n-1} + 12).\\
            &\implies a_n = 4(4^{n-1} - 4) + 12 = 4^n + 4.\\
            &\text{By induction, we have shown that }a_n = 4^n + 4\all n\in \N : n \geq 2. &\checkmark\\
            &\\
            &\text{(d) Explain why the following "proof" is not true}\\
            &\text{Suppose }x \text{ is a real number satisfying } (x-2)^3 + 3(x-2)^2 + 2x = 4.\\
            &\text{Then }x = 0 \lor 1.\\
            &\\
            &\text{answer: The proof (which is omitted) states that one should divide though by a defined variable }y,\\
            &\text{Which is not allowed. This is not allowed because when trying to find solutions,}\\
            &\text{we have eliminated some by the division operation. This means that there are potentially more solutions.}\\
            &\text{In this case, the proof misses the solution }x = 2. &\checkmark\\
        \end{align*}

    \section{Question Four}
        \begin{align*}
            &\\
            &\text{(a) Suppose }d, n \in \N.\text{ What does it mean for }d \mid n?\\
            &\text{answer: }d\mid n \impliedby k\in\N: n = dk. & \checkmark\\
            &\\
            &\text{(b) Suppose }p, q\text{ are prime and that }p \neq q. \text{how many divisors does } p^3 \times q^3\text{ have?}\\
            &\text{answer: }q^3\text{ has divisors }1, q, q^2, q^3 \text{ and } p^3\text{ has divisors }1, p, p^2, p^3.\\
            &\text{We may deduce that the number of divisors is the number of possible combinations or the divisors of }p \text{ and }q.\\
            &\text{This is }4^2 = 16\text{ divisors.} & \checkmark\\
            &\\
            &\text{(c) Use Euclid's algorithm to find }\gcd(198, 82).\\
            &\text{answer: Given }a = qb + r \implies 198 = 2(82) + 34,\\
            &\qquad 82 = 2(32) + 18,\\
            &\qquad 32 = 1(18) + 14,\\
            &\qquad 18 = 1(14) + 4,\\
            &\qquad 14 = 3(4) + 2,\\
            &\qquad 4 = 2(2) + 0 \implies \gcd(198, 82) = 2. &\checkmark\\
            &\\
            &\text{(d) Suppose }a, b\in\N \text{ and that } a\mid b. \text{ Prove that }a^2\mid b^2.\\
            &\text{answer: }a\mid b\implies b = ak\ a\in\N.~a^2 \mid b^2 \implies b^2 = a^2k,\\
            &k = \frac{b^2}{a^2} \implies b = a\frac{b^2}{a^2} \implies \frac{1}{b} = \frac{1}{a} \implies a = b & \xmark\\
            &\\
            &\quad\text{Real answer: Given }b = ak \text{ from }a\mid b\text{, we get }b^2 = a^2k^2, k^2\inn\qquad\square\\
            &\quad\text{The error arises from the fact that I used }b = ak \text{ and } b^2 = a^2k,\text{ but the k's are different.}\\
            &\\
            &\text{(e) Define the relation }\rel \text{ on }\N : a\rel b \impliedby a\mid 2b. \text{ Is }\rel \text{ transitive?}\\
            &\text{answer: A transitive relation is given by }a\rel b\text{ and } b \rel c \implies a\rel c.\\
            &\text{Let } 2b = ak_1 : a\rel b \AND\LET 2c = bk_2 : b \rel c.\\
            &a \rel c \impliedby a \mid 2c \text{ which yields } a\mid bk_2.\\
            &\text{Its already given that }bk_2\inn\implies \rel \text{ is transitive.}& \xmark\\
            &\\
            &\quad\text{Unsure as to why I stopped the proof/ disproof, }bk_2\text{ depends on } 2b=ak_1\text{ where } ak_1 \text{ is a multiple of 2,}\\
            &\quad\text{which may not be true. Therefore, }\rel \text{ is not transitive.}\\
        \end{align*}
    
    \section{Question Five}
        \begin{align*}
            &\text{Suppose }A \AND B \text{ are sets}: |A| = 6, |B| = 7, |A\cap B| = 3.\\
            &\\
            &\text{(a) Write down a 2 element subset of }B.\\
            &\text{answer: }\{\emptyset, B\} = B.&\xmark\\
            &\\
            &\quad\text{Real answer: "Write down the number of 2 element subsets of \textit{B}"}\\
            &\quad\text{Given this, } \binom{7}{2} = 21.\\
            &\\
            &\text{(b) Find }|A\cup B|\\
            &\text{answer: Given }|A| = 6, |B| = 7, |A\cap B| = 3, |A\cup B| = |A|-|A\cap B| + |B|-|A\cap B| = 7.&\xmark\\
            &\\
            &\quad\text{Real answer: }|A\cup B| = |A| + |A\not ~B| = 6 + (|B| - |A\cap B|) = 6 + 4 = 10.\\
            &\\
            &\text{(c) Let }D = \{C\in\power(A) : |C\cap B| = 1\}.\text{ Find }|D|.\\
            &\text{answer: }A = \{a_1, a_2, a_3, a_4, a_5, a_6\} : \power(A) = \{\{a_1\}, \{a_2\}, \dots, \{a_6\}, \dots, \{a_1, a_2, \dots, a_5, a_6\}\},\\
            &\text{Given }|A\cap B| = 4, |D| = 4.&\xmark\\
            &\\
            &\text{(d) Let }z = 1 + 3i.\\
            &\text{(i) }z^2 = (1+3i)(1+3i) = (6i - 8) & \checkmark\\
            &\text{(ii) }|z|^2 = (\sqrt{1^2 + 3^2})^2 = 10&\checkmark\\
            &\text{(iii) }w\inc: z-w \AND zw \inr:\\
            &\quad w = -1 + 3i: z - w = 1 + 3i - 1 + 3i = 6i,\\
            &\quad zw = (1 + 3i)(-1 + 3i) = -1 + 3i - 3i -9 = -10 \implies w = -1 + 3i.& \checkmark\\
        \end{align*}

\end{document}
