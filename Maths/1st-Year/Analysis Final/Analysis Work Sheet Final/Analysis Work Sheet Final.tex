\documentclass{article}
\usepackage{amsmath}
\usepackage{amssymb}
\usepackage{geometry}
\usepackage{microtype}
\hbadness=99999
\ProvidesPackage{mymathsymbols}[2024/07/28 A custom package for common mathematical symbols]

\RequirePackage{amsfonts} % This package is needed for \mathbb
\RequirePackage{pifont}   % This package is needed for \ding

% Define custom commands
\newcommand{\xmark}{\text{\ding{55}}}
\newcommand{\R}{\ensuremath{\mathbb{R}}}
\newcommand{\C}{\ensuremath{\mathbb{C}}}
\newcommand{\Z}{\ensuremath{\mathbb{Z}}}
\newcommand{\Q}{\mathbb{Q}}
\newcommand{\N}{\mathbb{N}}
\newcommand{\F}{\mathbb{F}}
\newcommand{\W}{\mathbb{W}}
\newcommand{\all}{~\forall~}
\newcommand{\power}{\mathcal{P}}

\endinput %Download the macros file locally onto machine.
\usepackage{array}
\usepackage{hyperref}
\hypersetup{
    colorlinks,
    citecolor=black,
    filecolor=black,
    linkcolor=black,
    urlcolor=black
}
\geometry{a4paper, margin=1.6in}

\title{Analysis Work Sheet Final}
\author{by Toby Chen}
\date{\today}

\begin{document}

    \maketitle

    \tableofcontents

    \section{Sequences}
        \sub{Q1}
            Find the formula for the $n$-th term of the sequence $1, -4, 9, -16, 25$.\\
            
            \ans Given $1, -4, 9, -16, 25$, we can clearly see that it's a sequence of squares with an alternating negative sign,\\

            \[a_n = (-1)^{n - 1}(n^2),\FOR n\geq 1.\]

        \sub{Q2}
            Find a formula for the nth term of the sequence in terms of $n$, where the sequence is $1,0,1,0,1,\dots$.\\

            \ans The sequence is a periodic one which oscillates from 1 to 0. Let us create a table of values to assist in this,

            \begin{center}
                \begin{tabular}{|c|c|c|c|c|c|}
                    \hline
                    1 & 2 & 3 & 4 & 5 & 6 \\
                    \hline
                    1 & 0 & 1 & 0 & 1 & 0 \\
                    \hline
                    odd & even & odd & even & odd & even\\
                    \hline
                \end{tabular}
            \end{center}

            We can see that when $n$ is even, we get a 0, and else is 1. This implies we will have a formula making use of $n/2$.\\

            Let $a_n = n/2 : a_n = 1/2, 1, 3/2, 2, 5/2, \dots$. It seems that we get closer to our desired result if we let $a_n = (n+1)/2 : a_n = 1, 3/2, 2, 5/2, 3, \dots$.\\

            I have just realized that the most suitable function would be $\sin(x)$, 

            \[a_n = \sin\left( \frac{n\pi}{2} \right)^2.\]

        \sub{Q3}
            Determine if the sequence $\left\{ a_n \right\}$ converges or diverges. Find the limit if the sequence converges. The sequence is $a_n = 4 + (0.3)^n$\\

            \ans 

            \begin{align*}
                &\lim_{n\rightarrow \infty} 4 + 0.3^n = 4\\
            \end{align*}

        \sub{Q4}
            Determine if the sequence $\left\{ a_n \right\}$ converges or diverges. Find the limit if the sequence converges. The sequence is $a_n = \left( \frac{n+6}{7n} \right)\left( 1 - \frac{6}{n} \right)$.\\

            \ans 

            \begin{align*}
                \text{Given }&a_n = \left( \frac{n+6}{7n} \right)\left( 1 - \frac{6}{n} \right),\\
                &\lim_{n \rightarrow \infty} \left( \frac{n+6}{7n} \right)\left( 1 - \frac{6}{n} \right) = \lim_{n \rightarrow \infty}\left( \frac{n+6}{7n} \right)\lim_{n \rightarrow \infty} \left( 1- \frac{6}{n} \right),\\
                &= \lim_{n\rightarrow\infty}\left( \frac{n+6}{7n} \right)(1) = \lim_{n\rightarrow\infty}\left( \frac{\frac{d}{dn}(n+6)}{\frac{d}{dn}(7n)} \right),\\
                &= \lim_{n\rightarrow\infty} \left( \frac{1}{7} \right) = \frac{1}{7}.\\
            \end{align*}

        \sub{Q5}
            Determine if the sequence $\left\{ a_n \right\}$ converges or diverges. Find the limit if the sequence converges. The sequence is $a_n = \sqrt[n]{4^n n}$.\\

            \ans 

            \begin{align*}
                &\lim_{n\rightarrow\infty}\sqrt[n]{4^n n} = L,\\
                &\ln(L) = \ln\left(\lim_{n\rightarrow\infty}(4^n n)^{1/n}\right) = \lim_{n\rightarrow\infty}\left( \ln\left( 4^nn \right)^{\frac{1}{n}} \right),\\
                &\ln(L) = \lim_{n\rightarrow\infty}\frac{1}{n} \ln\left( 4^n \right) + \ln(n) = \lim_{n\rightarrow\infty}\frac{1}{n}(n\ln(4) + \ln(n)),\\
                &\ln(L) = \frac{\lim_{n\rightarrow\infty}(n\ln(4) + \ln(n))}{n} = \lim_{n\rightarrow\infty}\ln(4) + \lim_{n\rightarrow\infty}\frac{\ln(n)}{n},\\
                &\ln(L) = \ln(4) + \lim_{n\rightarrow\infty}\frac{\frac{d}{dn}(\ln(n))}{\frac{d}{dn}(n)} = \ln(4) + \lim_{n\rightarrow\infty}\frac{1}{n},\\
                &\therefore \ln(L) = \ln(4) \implies L = 4.\\
            \end{align*}

        \sub{Q6}
            Use the definition of convergence to prove the given limit.

            \[\lim_{n\rightarrow\infty}\frac{\sin n}{n} = 0.\]

            \ans Let $\epsilon \in \R_+ : ~\exists~ N : n > N \implies |a_n - L| < \epsilon \FOR L$. Given the max of $|\sin n| = 1$, $L = 0 \because |a_n|<\epsilon ~\all n > \left\lceil\frac{1}{\epsilon}\right\rceil$. 

        \sub{Q7}
            Find a formula for the nth partial sum of the series and use it to find the series' sum if the series converges.\\

            The sum takes form,

            \[\frac{17}{1\times2} + \frac{17}{2\times3} + \frac{17}{3\times4} + \dots + \frac{17}{n(n+1)} + \dots\]

            \ans

            \begin{align*}
                &\frac{17}{1\times2} + \frac{17}{2\times3} + \frac{17}{3\times4} + \dots + \frac{17}{n(n+1)},\\
                =&17\left( \frac{1}{1\times2} + \frac{1}{2\times3} + \frac{1}{3\times4} + \dots + \frac{1}{n(n+1)} \right),\\
                =&\left( \frac{17}{(n+1)!} \right) = S_n.&\xmark\\
            \end{align*}

            \rans I made a very silly mistake and assumed the denominators were multiplied, though they obviously are not.

            \begin{align*}
                =&17\left( \frac{1}{1\times2} + \frac{1}{2\times3} + \frac{1}{3\times4} + \dots + \frac{1}{n(n+1)} \right),\\
            \end{align*}

            If we produce a table of values we may spot a pattern,

            \begin{center}
                \begin{tabular}{|c|c|c|c|c|c|c|}
                    \hline
                    $n$ & 1 & 2 & 3 & 4 & 5 & 6\\
                    \hline
                    $S_n$ & 17/2 & 34/3 & 51/4 & 68/5 & 85/6 & 102/7\\
                    \hline
                    Simplified & 17(1/2) & 17(2/3) & 17(3/4) & 17(4/5) & 17(5/6) & 17(6/7)\\
                    \hline
                \end{tabular}
            \end{center}

            We can now very easily see that,

            \[S_n = \frac{17n}{n+1}.\]

            Now to find the sum, we take the limit,

            \begin{align*}
                &\lim_{n\rightarrow\infty}\frac{17n}{n+1} = \lim_{n\tinf}\frac{17}{1} = 17.\\
            \end{align*}

        \sub{Q8}
            Determine if the geometric series converges or diverges. If a series converges, find its sum.\\
            
            \[\frac{1}{3} + \frac{1}{3}^2 + \frac{1}{3}^3 + \frac{1}{3}^4 + \dots\]

            \ans

            \begin{align*}
                &\frac{1}{3} + \frac{1}{3}^2 + \frac{1}{3}^3 + \frac{1}{3}^4 + \dots,\\
                =&\frac{1}{3} + \frac{1}{9} + \frac{1}{27} + \frac{1}{243} + \dots\\
                &\text{Lets try to find the }n\text{-th term, }a_n = \frac{1}{3^n}.\\
                &\text{Now we take the limit,}\\
                &\\
                &\lim_{n\tinf} \frac{1}{3^n} = 0.\\
            \end{align*}

            The series therefore does converge to zero. \xmark\\

            \rans We need either a formula for the $n$-th partial sum or we can use the fact that,

            \[\sum_{n = 1}^{\infty}ar^{n-1} = \frac{a}{1 - r},\FOR |r| < 1.\]

            In light of this, we can recall that a geometric series follows the pattern $a + ar + ar^2 + \dots$. We, in this case, let $a = 1/3$, and $r = 1/3$.\\

            $|r|< 1$ holds, so,

            \[S_{\infty} = \frac{1/3}{1 - 1/3} = 0.5.\]

        \sub{Q9}
            Use the $n$th-term test for divergence to show that the series is divergent, or state that the test is inconclusive.\\

            \[\sum_{n = 1}^{\infty} \cos \left( \frac{18}{n} \right).\]

            \ans Recall that the $n$-th term test asks,

            \[\IF \lim_{n\tinf} a_n \neq 0 \OR \lim_{n\tinf}a_n \text{ is undefined},~ \sum_{n = 1}^{\infty}a_n \text{ diverges}.\]

            In light of this, 

            \[\lim_{n\tinf} \cos \left( \frac{18}{n} \right) = 1 \implies \text{ Sum diverges}.\]

        \sub{Q10}
            Determine whether the series converges or diverges. If it converges, find its sum.

            \[\sum_{n = 0}^{\infty}e^{\frac{-5n}{2}}.\]

            \ans We can see that the series converges because the limit tends to 0. 

            \[\sum_{n = 0}^{\infty}e^{\frac{-5n}{2}} = \sum_{n = 0}^{\infty} \frac{1}{e^{\frac{5n}{2}}}.\]
            
            This series is indeed a geometric one, as shown, with $a = 1$, $r = 1/e^{\frac{5}{2}}$. This therefore means that,

            \[S_{\infty} = \frac{a}{1 - r} = \frac{1}{1 - \frac{1}{e^{\frac{5}{2}}}} = \frac{e^{\frac{5}{2}}}{e^{\frac{5}{2}} - 1}.\]

        \sub{Q11}



\end{document}