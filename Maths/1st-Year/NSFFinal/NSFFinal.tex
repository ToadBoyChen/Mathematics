\documentclass{article}
\usepackage{amsmath}
\usepackage{amssymb}
\usepackage{geometry}
\usepackage{array}
\geometry{a4paper, margin=1in}

\title{NSF Final}
\author{by Toby Chen}
\date{\today}

\begin{document}

    \maketitle

    \section{Notation}

        \subsection{Summation}
            The summation is a mathematical operation that adds up elements up to, and including a specified index.

            \[\sum_{n = 0}^{m} f(n) = N\]

            In most important cases we work with $m = \infty$.

        \subsection{Product}
            The product is a mathematical operation that multiplies up elements up to, and including a specified  index.

            \[\prod_{n = 1}^{m} f(n) = N\]

            It may make sense depending on $f(n)$ to start the product at $n=1$.

            \subsubsection{Factorial}
                The factorial is defined as,

                \[m! = \prod_{n = 1}^{m}n\]

    \section{Logic}
        \subsection{A Statement}
            A statement is a sentence which may exist on its own and is either true or false.

        \subsection{Negation}
            If $P$ is a statement, the negation of $P$ is not $P$, or that of is $P$ is false.\\

            Take the example $x=2$. The negation is $x\neq 2$.

        \subsection{Implication}
            Suppose $P$ and $Q$ are statements. 'if $P$, then $Q$' is an implication, which is written $P \implies Q$.\\

            We have a special case where if $P \implies Q$, and $P$ is true, then $Q$ may not be false. One should note that the implication, unlike our spoken language, does not imply a direction causation, e.g., 'I go to sleep $\implies$ Donald Trump assassination attempt'. It should be clear that there is no causation, however mathematically this statement holds.\\

            An important table we should therefore observe is as follows,

            \begin{center}
                \begin{tabular}{|c|c|c|}
                    \hline
                    $P$ & $Q$ & $P \implies Q$ \\
                    \hline
                    true & true & true \\
                    true & false & false \\
                    false & true & true \\
                    false & false & true \\
                    \hline
                \end{tabular}
            \end{center}

        \subsection{Quantifiers}
            There are two special mathematical objects that allow us to construct statements involving a variable in a 'general statement'.\\

            We can write '$n$ is even', however is has the implicit dependency on $n\in\mathbb{Z}$, so we must state this explicitly, '$n$ is even $\forall ~n \in \mathbb{Z}~$'. The symbol $\forall$, literally means 'for all', and is a universal quantifier. Note that the statement we constructed is false. 

        \subsection{Converse and Contrapositive}
            The converse of $P\implies Q$ is given by $Q \implies P$. The contrapositive of $P \implies Q$ is given by $\bar{Q} \implies \bar{P}$. \\

            One should note that the contrapositive is the exact same thing as the normal statement, just 'rephrased'. 

            \begin{center}
                \begin{tabular}{|c|c|c|c|c|c|}
                    \hline
                    $P$ & $Q$ & $P \implies Q$ & $\bar{P}$ & $\bar{Q}$ & $\bar{Q} \implies \bar{P}$ \\
                    \hline
                    true & true & true & false & false & true \\
                    true & false & false & false & true & false \\
                    false & true & true & true & false & true \\
                    false & false & true & true & true & true \\
                    \hline
                \end{tabular}
            \end{center}

    \section{Proof}
         A proof is a logical argument that states a fact with 100\% certainty.\\

         The first part of a proof sets the context. The second part creates statements which are linked through implications. We also make assumptions throughout, like '$n$ is even'. The last part is the conclusion.\\
         
         It maybe that to prove $P\implies Q$, we prove $P$, though to prove $P \iff Q$, one must prove $P$ and $Q$.\\

         A special proof technique is proof via contrapositive. 

    \section{Analysis}

\end{document}
