\documentclass{article}
\usepackage{amsmath}
\usepackage{amssymb}
\usepackage{geometry}
\ProvidesPackage{mymathsymbols}[2024/07/28 A custom package for common mathematical symbols]

\RequirePackage{amsfonts} % This package is needed for \mathbb
\RequirePackage{pifont}   % This package is needed for \ding

% Define custom commands
\newcommand{\xmark}{\text{\ding{55}}}
\newcommand{\R}{\ensuremath{\mathbb{R}}}
\newcommand{\C}{\ensuremath{\mathbb{C}}}
\newcommand{\Z}{\ensuremath{\mathbb{Z}}}
\newcommand{\Q}{\mathbb{Q}}
\newcommand{\N}{\mathbb{N}}
\newcommand{\F}{\mathbb{F}}
\newcommand{\W}{\mathbb{W}}
\newcommand{\all}{~\forall~}
\newcommand{\power}{\mathcal{P}}

\endinput %Download the macros file locally onto machine.
\usepackage{array}
\usepackage{hyperref}
\hypersetup{
    colorlinks,
    citecolor=black,
    filecolor=black,
    linkcolor=black,
    urlcolor=black
}
\geometry{a4paper, margin=1.6in}

\title{Probability and Statistics January Exam 2024}
\author{by Toby Chen}
\date{\today}

\begin{document}

    \maketitle

    \tableofcontents

    \section{Question One}
        Assume the number of buses passing through a certain stop in 1 hour follows a Poisson distribution with average $\lambda= 5.5$. Calculate the probability that at least one bus turns up in 1 hour.\\

        \textit{Answer}: Recall that the Poisson distribution is (?). \xmark\\

        \textit{Real Answer}: the Poisson distribution given $\bar{x} = \mu = \E(X) = \lambda$, and the $\po(X = k)$ is given by

        \[\frac{\lambda^ke^{-\lambda}}{k!}.\]

        One should consider the fact that the events are assumed to be independent. In light of this, given $\lambda = 5.5$ and $\po(X \geq 1) = 1 - \po(X = 0)$, we get,

        \[1 - \frac{5.5^0 e^{-5.5}}{0!} = 0.995913... \approx 0.995,~ 99.5\%\]

    \section{Question Two}
        You roll a fair dice once

        \subsection{Q2 a}
            Calculate the probability of getting the number six.\\

            \textit{Answer}:

            \[\po(X = 6) = \frac{|\{6\}|}{|\Omega|} = \frac{|\{6\}|}{|\{1,2,3,4,5,6\}|} = \frac{1}{6}.~\checkmark\]

        \subsection{Q2 b}
            You roll a fair dice n times. Calculate the probability of getting at least one six.\\

            \textit{Answer}: To begin our investigation, let $n = 1 : \po(X = 6) = \frac{1}{6},~ n=2 : \po(X = 2) = \frac{1}{36},~ n=3: \po(X = 6) = \frac{1}{216}$, etc. We can clearly see a climbing in the powers of six where the index linearly increases with $n$.\\ 
            
            We may therefore deduce that for $n$-times, $\po(X=6) = 6^{-n}$ \xmark.\\

            \textit{Real answer}: I simply misunderstood the question. They ask at least one six, implying that there maybe more than 1 six. Therefore, we don't want to calculate $\po(X = 6)$, as It's actually not relevant to the question. We want to calculate $\po(\text{At least 1 six})$. We could cleverly set $Y =$ (number of sixes) and then state $\po(Y \geq 1) = 1 - \po(Y = 0)$.\\

            \[1 - \po(Y = 0) = 1 - \left( \frac{5}{6} \right)^n.\]

        \subsection{Q2 c}
            Find the values of $n$ such that this probability is greater than or equal to $50\%$.\\

            \textit{Answer}: given our formula $1 - \left( \frac{5}{6} \right)^n$, we want,

            \begin{align*}
                1 - \left( \frac{5}{6} \right)^n &\geq 0.5,\\
                \left( \frac{5}{6} \right)^n&\geq 0.5,\\
                \ln\left( \left( \frac{5}{6} \right)^n \right)&\geq \ln(0.5),\\
                n&\geq\frac{\ln(0.5)}{\ln\left( \left( \frac{5}{6} \right)\right)},\\
                n&\geq3.8016 \approx 3.8 \approx 4. ~\checkmark
            \end{align*}

    \section{Question Three}
        Given the random variables $X$ and $Y$ with pmf of,

        \begin{align*}
            \po(X = k)& = \begin{cases}
                \frac{1}{2},&k = 0,\\
                \frac{1}{2},&k = 3,\\
                0,&k\notin\{0, 3\}.\\
            \end{cases}\\
            &\\
            \po(Y = k)& = \begin{cases}
                \frac{1}{3},&k=-1,\\
                \frac{2}{3},&k=1,\\
                0,&k\notin\{-1, 1\}.\\
            \end{cases}\\
        \end{align*}

        \subsection{Q3 a}
            Provide an example of a joint probability distribution for $X$ and $Y$ such that $X$ and $Y$ are independent. You can use a table as below:

            \begin{center}
                \begin{tabular}{|c|c|c|c|}
                    \hline
                    & $X = 0$ & $X = 3$ & marginal\\
                    \hline
                    $Y = -1$ & ? & ? & $\po(Y = -1) = 1/3$\\
                    \hline
                    $Y = 1$ & ? & ? & $\po(Y = 1) = 2/3$\\
                    \hline
                    marginal & $\po(X = 0) = 1/2$ & \po(X = 3) = 1/2 & \\
                    \hline
                \end{tabular}
            \end{center}

            \textit{Answer}: 

            \begin{center}
                \begin{tabular}{|c|c|c|c|}
                    \hline
                    & $X = 0$ & $X = 3$ & marginal\\
                    \hline
                    $Y = -1$ & $\po(Y = -1)\cap\po(X = 0)$ & $\po(Y = -1)\cap\po(X = 3)$ & $\po(Y = -1) = 1/3$\\
                    \hline
                    $Y = 1$ & $\po(Y = 1)\cap\po(X = 0)$ & $\po(Y = 1)\cap\po(X = 3)$ & $\po(Y = 1) = 2/3$\\
                    \hline
                    marginal & $\po(X = 0) = 1/2$ & \po(X = 3) = 1/2 & \\
                    \hline
                \end{tabular}\\
                $\downarrow$
            \end{center}

            \begin{center}
                \begin{tabular}{|c|c|c|c|}
                    \hline
                    & $X = 0$ & $X = 3$ & marginal\\
                    \hline
                    $Y = -1$ & $1/3 \times 1/2$ & $1/3 \times 1/2$ & $\po(Y = -1) = 1/3$\\
                    \hline
                    $Y = 1$ & $2/3 \times 1/2$ & $2/3 \times 1/2$ & $\po(Y = 1) = 2/3$\\
                    \hline
                    marginal & $\po(X = 0) = 1/2$ & \po(X = 3) = 1/2 & \\
                    \hline
                \end{tabular}\\
                $\downarrow$
            \end{center}

            \begin{center}
                \begin{tabular}{|c|c|c|c|}
                    \hline
                    & $X = 0$ & $X = 3$ & marginal\\
                    \hline
                    $Y = -1$ & $1/6$ & $1/6$ & $\po(Y = -1) = 1/3$\\
                    \hline
                    $Y = 1$ & $1/3$ & $1/3$ & $\po(Y = 1) = 2/3$\\
                    \hline
                    marginal & $\po(X = 0) = 1/2$ & \po(X = 3) = 1/2 & \\
                    \hline
                \end{tabular} \checkmark
            \end{center}

        \subsection{Q3 b}
            Provide an example of a joint probability distribution for $X$ and $Y$ such that $X$ and $Y$ are not independent.

            \begin{center}
                \begin{tabular}{|c|c|c|c|}
                    \hline
                    & $X = 0$ & $X = 3$ & marginal\\
                    \hline
                    $Y = -1$ & $1/6 + 0.01$ & $1/6 + 0.01$ & $\po(Y = -1) = 1/3$\\
                    \hline
                    $Y = 1$ & $1/3 + 0.01$ & $1/3 + 0.01$ & $\po(Y = 1) = 2/3$\\
                    \hline
                    marginal & $\po(X = 0) = 1/2$ & \po(X = 3) = 1/2 & \\
                    \hline
                \end{tabular} \checkmark
            \end{center}

    \section{Question Four}
        The average size of a foreign exchange transaction in a certain trading desk is $\$25.1m$ with a standard deviation of $\$15m$. We assume these numbers follow a normal distribution. We are concerned about the impact of a new regulatory reporting procedure that kicks in for transactions larger than $\$50m$. Calculate the probability that a given transaction will be larger than $\$50m$.\\

        \textit{Answer}: Given that the distribution of the transactions follows a normal distribution, $\mathcal{N}(\mu,\sigma^2)$, we are concerned with the pdf of,

        \[f(x) = \frac{1}{\sqrt{2\pi\sigma^2}}e^{-\frac{(x-\mu)^2}{2\sigma^2}}.\]

        We want to find, quote, "the probability that a given transaction will be larger than $\$50m$", which translates mathematically to $\po(X > 50m) = 1 - \po(X \leq 50m)$.\\

        We can use therefore, the cdf of $\mathcal{N}(\mu,\sigma^2)$ to find $\po(X \leq 50m)$, which is, 

        \[\Phi\left( \frac{x - \mu}{ \sigma} \right).\]

        We therefore want to calculate $\Phi((50m - 25.1m)/(15m)) = \Phi(1.66)$. One must look this value up in the given table of values, where one obtains $0.9515$.\\

        Given that $\po(X \leq 50) = 0.9515$, $\po(X > 50m) = 1-0.9515 = 0.0485 \approx 0.05$. The chance of a transaction being larger than $\$50m$ is $5\%$. \checkmark\checkmark\checkmark

    \section{Question Five}
        This question demonstrates a special feature of the exponential distribution that makes it suitable for applications to waiting times. It is called the memoryless property. This is a characteristic of the Exponential random variable that might or might not correspond to what happens in real-life applications.\\

        Assume that $X\sim\text{Exp}(\lambda)$ models the wait time in hours to the next bus if on average there are $\lambda$ busses per hour.

        \subsection{Q5 a}
            $x_1\in\R_+$, what is the probability that no buses arrived before time $x_1$, i.e., calculate $\po(X > x_1) = 1 - \po(X \leq x_1)$.\\

            \textit{Answer}: The cdf, as needed, of the exponential distribution is $1 - e^{-\lambda x}$.\\

            We are working with the general $x_1$, and with then general $\lambda$, so $1 - (1-e^{\lambda x_1}) = 1 - 1 + e^{\lambda x_1} = e^{\lambda x_1}$. \xmark\\

            \textit{Real answer}: Forgot the negative sign for lambda, $e^{-\lambda x_1}$.

        \subsection{Q5 b}
            For $0 < x_1 < x_2$, calculate the probability that the bus will arrive between the two, i.e., $\po(x_1 < X < x_2)$.\\

            \textit{Answer}: $\po(x_1 < X < x_2) = \po(X > x_1) + \po(X < x_2) = (1-\po(X \leq x_1)) + \po(X < x_2)$.\\

            Once again referring to the cdf, that being $1 - e^{-\lambda x}$, we get the following,

            \[\left( 1 - (1 - e^{-\lambda x_1}) \right) + (1 - e^{-\lambda x_2}) = 1 + e^{-\lambda x_1} - e^{-\lambda x_2}.~\xmark\]

            \textit{Real answer}: In fact we may use some basic logic to state that $\po(x_1 < X < x_2) = \text{cdf}(x_2) - \text{cdf}(x_1)$. Following this, we get $(1 - e^{-\lambda x_2}) - (1 - e^{-\lambda x_1}) = e^{-\lambda x_1} - e^{-\lambda x_2}$.

        \subsection{Q5 c}
            Calculate $\po(X < x_2 | X > x_1)$.\\

            \textit{Answer}: $\po(X < x_2 | X > x_1) = (\po(X < x_2)\cap \po(X > x_1))/(\po(X < x_2) \cup \po(X > x_1))$. This yields, 

    \section{Question Six}

    \section{Question Seven}

\end{document}