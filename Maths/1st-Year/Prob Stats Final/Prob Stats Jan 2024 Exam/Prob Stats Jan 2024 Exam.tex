\documentclass{article}
\usepackage{amsmath}
\usepackage{amssymb}
\usepackage{geometry}
\ProvidesPackage{mymathsymbols}[2024/07/28 A custom package for common mathematical symbols]

\RequirePackage{amsfonts} % This package is needed for \mathbb
\RequirePackage{pifont}   % This package is needed for \ding

% Define custom commands
\newcommand{\xmark}{\text{\ding{55}}}
\newcommand{\R}{\ensuremath{\mathbb{R}}}
\newcommand{\C}{\ensuremath{\mathbb{C}}}
\newcommand{\Z}{\ensuremath{\mathbb{Z}}}
\newcommand{\Q}{\mathbb{Q}}
\newcommand{\N}{\mathbb{N}}
\newcommand{\F}{\mathbb{F}}
\newcommand{\W}{\mathbb{W}}
\newcommand{\all}{~\forall~}
\newcommand{\power}{\mathcal{P}}

\endinput %Download the macros file locally onto machine.
\usepackage{array}
\usepackage{hyperref}
\hypersetup{
    colorlinks,
    citecolor=black,
    filecolor=black,
    linkcolor=black,
    urlcolor=black
}
\geometry{a4paper, margin=1.6in}

\title{Probability and Statistics January Exam 2024}
\author{by Toby Chen}
\date{\today}

\begin{document}

    \maketitle

    \tableofcontents

    \section{Question One}
        Assume the number of buses passing through a certain stop in 1 hour follows a Poisson distribution with average $\lambda= 5.5$. Calculate the probability that at least one bus turns up in 1 hour.\\

        \textit{Answer}: Recall that the Poisson distribution is (?). \xmark\\

        \textit{Real Answer}: the Poisson distribution given $\bar{x} = \mu = \E(X) = \lambda$, and the $\po(X = k)$ is given by

        \[\frac{\lambda^ke^{-\lambda}}{k!}.\]

        One should consider the fact that the events are assumed to be independent. In light of this, given $\lambda = 5.5$ and $\po(X \geq 1) = 1 - \po(X = 0)$, we get,

        \[1 - \frac{5.5^0 e^{-5.5}}{0!} = 0.995913... \approx 0.995,~ 99.5\%\]

    \section{Question Two}
        You roll a fair dice once

        \subsection{Q2 a}
            Calculate the probability of getting the number six.\\

            \textit{Answer}:

            \[\po(X = 6) = \frac{|\{6\}|}{|\Omega|} = \frac{|\{6\}|}{|\{1,2,3,4,5,6\}|} = \frac{1}{6}.~\checkmark\]

        \subsection{Q2 b}
            You roll a fair dice n times. Calculate the probability of getting at least one six.\\

            \textit{Answer}: To begin our investigation, let $n = 1 : \po(X = 6) = \frac{1}{6},~ n=2 : \po(X = 2) = \frac{1}{36},~ n=3: \po(X = 6) = \frac{1}{216}$, etc. We can clearly see a climbing in the powers of six where the index linearly increases with $n$.\\ 
            
            We may therefore deduce that for $n$-times, $\po(X=6) = 6^{-n}$ \xmark.\\

            \textit{Real answer}: I simply misunderstood the question. They ask at least one six, implying that there maybe more than 1 six. Therefore, we don't want to calculate $\po(X = 6)$, as It's actually not relevant to the question. We want to calculate $\po(\text{At least 1 six})$. We could cleverly set $Y =$ (number of sixes) and then state $\po(Y \geq 1) = 1 - \po(Y = 0)$.\\

            \[1 - \po(Y = 0) = 1 - \left( \frac{5}{6} \right)^n.\]

        \subsection{Q2 c}
            Find the values of $n$ such that this probability is greater than or equal to $50\%$.\\

            \textit{Answer}: given our formula $1 - \left( \frac{5}{6} \right)^n$, we want,

            \begin{align*}
                1 - \left( \frac{5}{6} \right)^n &\geq 0.5,\\
                \left( \frac{5}{6} \right)^n&\geq 0.5,\\
                \ln\left( \left( \frac{5}{6} \right)^n \right)&\geq \ln(0.5),\\
                n&\geq\frac{\ln(0.5)}{\ln\left( \left( \frac{5}{6} \right)\right)},\\
                n&\geq3.8016 \approx 3.8 \approx 4. ~\checkmark
            \end{align*}

            

\end{document}